
%
%	Appendix
%

\section{Neutrino oscillation between two neutrino species in vacuum}\label{AppA}
\Headerfooter{Neutrino oscillation between two neutrino species in vacuum}
\vs\hs Here the neutrino oscillation between two neutrino species in vacuum is considered.
The relationship between flavor eigenstates and mass eigenstates can be expressed as
\begin{eqnarray}\label{App_Eq_FlavMass}
	\left(
	\begin{array}{c}
		\ket{\nu_{\alpha}}\\
		\ket{\nu_{\beta}}
	\end{array}
	\right)
	=\left(
	\begin{array}{cc}
		\cos\theta&\sin\theta\\
		-\sin\theta&\cos\theta
	\end{array}
	\right)
	\left(
	\begin{array}{c}
		\ket{\nu_{1}}\\
		\ket{\nu_{2}}
	\end{array}
	\right)
	=\left(
	\begin{array}{c}
		\cos\theta\ket{\nu_{1}}+\sin\theta\ket{\nu_{2}}\\
		-\sin\theta\ket{\nu_{1}}+\cos\theta\ket{\nu_{2}}
	\end{array}
	\right).
\end{eqnarray}
From Equation~(\ref{App_Eq_FlavMass}),
\begin{eqnarray}
	\left(
	\begin{array}{c}
		\ket{\nu_{1}}\\
		\ket{\nu_{2}}
	\end{array}
	\right)
	=\left(
	\begin{array}{c}
		\cos\theta\ket{\nu_{\alpha}}-\sin\theta\ket{\nu_{\beta}}\\
		\sin\theta\ket{\nu_{\alpha}}+\cos\theta\ket{\nu_{\beta}}
	\end{array}
	\right).
\end{eqnarray}


\newpage
