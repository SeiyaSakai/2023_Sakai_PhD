
%	Abstract

\vspace*{\stretch{1}}
\begin{center}
	{\LARGE\bfseries ABSTRACT}
\end{center}
\vs\hs
In July 2020, 0.011\% by mass of gadolinium (Gd) was loaded in the Super-Kamiokande (SK) detector to improve the neutron-tagging efficiency, and the SK-Gd experiment started.
The SK-Gd experiment is aiming to achieve the world's first observation of the diffuse supernova neutrino background (DSNB).
One of the main backgrounds in the SK-Gd DSNB search is the atmospheric neutrino-oxygen neutral-current quasielastic scattering (NCQE) reactions.
In order to estimate the NCQE events precisely, we must understand the number and energy of deexcitation gamma-rays and the number of neutrons by the nucleon-nucleus interactions in water (secondary interactions).
So far, the secondary interaction model based on the Bertini Cascade model (BERT) was the only choice in the SK detector simulation.
However, now other secondary interaction models like the Binary Cascade model (BIC) and the Li$\grave{\text{e}}$ge Intranuclear Cascade model (INCL++) are available.\\
\hs
Using 552.2 days of data from August 2020 to June 2022, we performed the comparison of secondary interaction models using atmospheric neutrino events and the measurement of the atmospheirc neutrino-oxygen NCQE cross section in the energy range from 160~MeV to 10~GeV.
First, we compared the distributions of reconstructed Cherenkov angle, visible energy, and the number of delayed signals for the three secondary interaction models.
The results suggest that the evaporation model used in BIC and INCL++ reproduces the observed data better than that used in BERT for all distributions.
Moreover, we measure the NCQE cross section to be $0.74 \pm 0.22({\rm stat.})^{+0.85}_{-0.15}({\rm syst.}) \times 10^{-38}\,{\rm cm}^{2}/{\rm oxygen}$ in the energy range from 160~MeV to 10~GeV, which is consistent with the atmospheric neutrino-flux-averaged theoretical NCQE cross section and the measurement in the SK pure-water phase within the uncertainties.\\
\hs
Now we continue the observation with a 0.03\% Gd-loaded SK detector, the phase known as SK-VII.
By combining about three years of data in SK-VII, the statistical uncertainty will be half of this work, and the secondary interaction models will be able to be verified more precisely.
Furthermore, the evaporation model can be determined at 5$\sigma$ by combining about four years of data in SK-VII.
Once the evaporation model is determined, the systematic uncertainty of measured NCQE cross section is significantly reduced.
\vspace{\stretch{2}}
\thispagestyle{empty}
\clearpage
